\documentclass{article}

\usepackage{graphicx} % Required for inserting images
\usepackage[left=1in,right=1in,top=1in,bottom=1in]{geometry} \usepackage{amsmath}
\usepackage{amsthm} %proof environment
\usepackage{amsthm} %proof environment
\usepackage{amssymb}
\usepackage{amsfonts}
\usepackage{enumitem} %nice lists
\usepackage{verbatim} %useful for something 
\usepackage{xcolor}
\usepackage{setspace}
\usepackage{titlesec}
\usepackage{blindtext} % I have no idea what this is 
\usepackage{caption}  % need this for unnumbered captions/figures
\usepackage{natbib}
\usepackage{appendix}
\usepackage{tikz}
\usepackage{hyperref}


\hypersetup{
    colorlinks=true,
    linkcolor=blue,
    filecolor=magenta,      
    urlcolor=blue,
    pdftitle={Overleaf Example},
    pdfpagemode=FullScreen,
    }

\titleformat{\section}{\bfseries\Large}{Problem \thesection:}{5pt}{}

\begin{document}

\title{AM 216 - Stochastic Differential Equations: Final Exam}
\author{Dante Buhl}


\newcommand{\wrms}{w_{\text{rms}}}
\newcommand{\bs}[1]{\boldsymbol{#1}}
\newcommand{\tb}[1]{\textbf{#1}}
\newcommand{\bmp}[1]{\begin{minipage}{#1\textwidth}}
\newcommand{\emp}{\end{minipage}}
\newcommand{\R}{\mathbb{R}}
\newcommand{\C}{\mathbb{C}}
\newcommand{\N}{\mathcal{N}}
\newcommand{\Var}{\text{Var}}
\newcommand{\Cov}{\text{Cov}}
\newcommand{\Bino}{\text{Bino}}
\newcommand{\Norm}{\mathcal{N}}
\newcommand{\erf}{\text{erf}}
%\newcommand{\K}{\bs{\mathrm{K}}}
\newcommand{\m}{\bs{\mu}_*}
\newcommand{\s}{\bs{\Sigma}_*}
\newcommand{\dt}{\Delta t}
\newcommand{\dx}{\Delta x}
\newcommand{\tr}[1]{\text{Tr}(#1)}
\newcommand{\Tr}[1]{\text{Tr}(#1)}
\newcommand{\Div}{\nabla \cdot}
\renewcommand{\div}{\nabla \cdot}
\newcommand{\Curl}{\nabla \times}
\newcommand{\Grad}{\nabla}
\newcommand{\grad}{\nabla}
\newcommand{\grads}{\nabla_s}
\newcommand{\gradf}{\nabla_f}
\newcommand{\xs}{x_s}
\newcommand{\x}{\bs{x}}
\newcommand{\xf}{x_f}
\newcommand{\ts}{t_s}
\newcommand{\tf}{t_f}
\newcommand{\pt}{\partial t}
\newcommand{\pz}{\partial z}
\newcommand{\uvec}{\bs{u}}
\newcommand{\bvec}{\bs{B}}
\newcommand{\nvec}{\hat{\bs{n}}}
\newcommand{\tu}{\tilde{\uvec}}
\newcommand{\B}{\bs{B}}
\newcommand{\A}{\bs{A}}
\newcommand{\jvec}{\bs{j}}
\newcommand{\F}{\bs{F}}
\newcommand{\T}{\tilde{T}}
\newcommand{\ez}{\bs{e}_z}
\newcommand{\ex}{\bs{e}_x}
\newcommand{\ey}{\bs{e}_y}
\newcommand{\eo}{\bs{e}_{\bs{\Omega}}}
\newcommand{\ppt}[1]{\frac{\partial #1}{\partial t}}
\newcommand{\pp}[2]{\frac{\partial #1}{\partial #2}}
\newcommand{\pptwo}[2]{\frac{\partial^2 #1}{\partial #2^2}}
\newcommand{\ddtwo}[2]{\frac{d^2 #1}{d #2^2}}
\newcommand{\DDt}[1]{\frac{D #1}{D t}}
\newcommand{\ppts}[1]{\frac{\partial #1}{\partial t_s}}
\newcommand{\pptf}[1]{\frac{\partial #1}{\partial t_f}}
\newcommand{\ppz}[1]{\frac{\partial #1}{\partial z}}
\newcommand{\ddz}[1]{\frac{d #1}{d z}}
\newcommand{\ppzetas}[1]{\frac{\partial^2 #1}{\partial \zeta^2}}
\newcommand{\ppzs}[1]{\frac{\partial #1}{\partial z_s}}
\newcommand{\ppzf}[1]{\frac{\partial #1}{\partial z_f}}
\newcommand{\ppx}[1]{\frac{\partial #1}{\partial x}}
\newcommand{\ddx}[1]{\frac{d #1}{d x}}
\newcommand{\ppxi}[1]{\frac{\partial #1}{\partial x_i}}
\newcommand{\ppxj}[1]{\frac{\partial #1}{\partial x_j}}
\newcommand{\ppy}[1]{\frac{\partial #1}{\partial y}}
\newcommand{\ppzeta}[1]{\frac{\partial #1}{\partial \zeta}}
\renewcommand{\k}{\bs{k}}
\newcommand{\real}[1]{\text{Re}\left[#1\right]}


\maketitle 
% This line removes the automatic indentation on new paragraphs
\setlength{\parindent}{0pt}

\section{}
    \begin{enumerate}[label=\roman*)]
        \item 
            \begin{align*}
                Y &= \int_0^1\cos(\pi s)dW(s)
                \\
                &= \int_0^1\cos(\pi s)\sqrt{dt}X(s)
                \\
                &\sim \mathcal{N}\left(0, \int_0^1\cos^2(\pi s)dt\right)
                \\
                &\sim \mathcal{N}\left(0, \frac{1}{2}\int_0^11 + \cos(2\pi s)dt\right)
                \\
                &\sim \mathcal{N}\left(0,\frac{1}{2}\right)
                \\
                E(Y) &= 0
                \\
                \Var(Y) &= \frac{1}{2}
            \end{align*}
        \item
            Since $Y$ is gaussian, its PDF can be rewritten simply
            using,
            \begin{align*}
                \rho_Y(y) &=
                \frac{1}{\sigma_Y\sqrt{2\pi}}\exp\left(-\frac{y^2}{2 
                \sigma_Y^2}\right)
                \\
                &= \frac{1}{2\sqrt{\pi}}\exp\left(-y^2\right)
            \end{align*}
    \end{enumerate}
\section{}
    \begin{align*}
        \frac{dX}{X} &= dt + 2dW
        \\
        \ln|X| &= \exp(t + 2W(t))
        \\
        X(t) &= X_0\exp(t + 2W(t))
        \\
        &= 3\exp(t + 2W(t))
    \end{align*}
\section{}
    We begin by rephrasing this as a brownian bridge problem
    \begin{align*}
        I &= \int_0^TW^2(t)dt
        \\
        E(I|W(T) = w_1) &= \int_0^TE(W^2(t) | W(T) = w_1)dt
        \\
        &= \int_0^TE(B^2(t))dt
        \\
        &= \int_0^TE\left(W^2(t) + \frac{t_i^2}{T^2}\left(w_1 -
        W(T)\right)^2 - \frac{2t_i}{T}W(t)\left(w_1 - W(T)\right)\right)dt
        \\
        &= \int_0^T t + \frac{t^2}{T^2}E\left(w_1^2 + W^2(T) -2
        W(T)w_1\right) + \frac{2t}{T}E(W(t)W(T)) dt
        \\
        &= \int_0^T t + \frac{t^2}{T^2}\left(w_1^2 + T\right) 
        + \frac{2t^2}{T} dt
        \\
        &= \frac{1}{2}T^2 + \frac{w_1^2T}{3} + \frac{T^2}{3} + \frac{2T^2}{3}
        \\
        &= \frac{3}{2}T^2 + \frac{w_1^2}{3}T
    \end{align*}
\section{}
    \begin{enumerate}[label=\roman*)]
        \item 
            \begin{align*}
                E(B(t)) &= t^{3/2}E(W(1/t^2)
                \\
                &= 0
                \\
                \Var(B(t)) &= t^3\Var(W(1/t^2))
                \\
                &= t^3\frac{1}{t^2}
                \\
                &= t
            \end{align*}
        \item 
            In order to show that these are not the same stochastic process it
            sufficies to show that their SDEs are not the same. We have for
            example, 
            \begin{align*}
                dW(t) &= dW(t)
                \\
                dB(t) &= B(t + dt) - B(t)
                \\
                &= (t + dt)^{3/2}W\left(\frac{1}{(t + dt)^2}\right) -
                t^{3/2}W(1/t^2)
                \\
                &= \left(t^{3/2} + \frac{3dt}{2}t^{1/2} +
                h.o.t.\right)W\left(\frac{1}{(t + dt)^2}\right) - t^{3/2}W(1/t^2)
                \\
                dt' &= \frac{1}{t^2} - \frac{1}{(t + dt)^2}
                \\
                &= \frac{2dt}{t^2(t + dt)}
                \\
                dB(t) &= 
                \left(t^{3/2} + \frac{3dt}{2}t^{1/2} +
                h.o.t.\right)\left(W(1/t^2) + \sqrt{dt'}\mathcal{N}(0,1)\right) - t^{3/2}W(1/t^2)
                \\
                &= \frac{3dt}{2}t^{1/2}W(1/t^2) + t^{3/2}dW'
                \\
                &= \frac{3B(t)}{2t}dt + t^{3/2}dW'
                \\
                &\neq dW
            \end{align*}
    \end{enumerate}
\section{}
    \begin{enumerate}[label=\roman*)]
        \item 
            We have the given BVP for $T$ is given by the following form,
            \begin{align*}
                &\begin{cases}\frac{a(x)}{2}T_{xx} + b(x)T_x = -1
                \\
                T(\varepsilon) = 0 \quad T'(1) = 0\end{cases}
                \\
                &\begin{cases}\frac{9x^{4/3}}{2}T_{xx} + 3x^{1/3}T_x = -1
                \\
                T(\varepsilon) = 0 \quad T'(1) = 0\end{cases}
            \end{align*}
        \item
            Solving this BVP is a matter of the method of integrating factor. 
            \begin{align*}
                T_{xx} + \frac{2}{3x}T_x &= -\frac{2}{9}x^{-4/3}
                \\
                \mu(x) &= x^{2/3}
                \\
                T_x &= -\frac{2}{9}x^{-2/3}\int x^{-2/3}dx
                \\
                &= -\frac{2}{9}x^{-2/3}(3x^{1/3} + C_1)
                \\
                &= -\frac{2}{3}x^{-1/3} + C_1x^{-2/3}
                \\
                T(x) &= -x^{2/3} + C_1x^{1/3} + C_2
                \\
                T'(1) &= -\frac{2}{3} + \frac{C_1}{3} = 0
                \\
                T(\varepsilon) &= -\varepsilon^{2/3} + 2\varepsilon^{1/3} + C_2=
                0 
                \\
                T(x) &= -x^{2/3} + 2x^{1/3} + \varepsilon^{2/3} -
                2\varepsilon^{1/3}
            \end{align*}
    \end{enumerate}

\end{document}
