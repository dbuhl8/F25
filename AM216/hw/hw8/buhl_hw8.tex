\documentclass{article}

\usepackage{graphicx} % Required for inserting images
\usepackage[left=1in,right=1in,top=1in,bottom=1in]{geometry} \usepackage{amsmath}
\usepackage{amsthm} %proof environment
\usepackage{amsthm} %proof environment
\usepackage{amssymb}
\usepackage{amsfonts}
\usepackage{enumitem} %nice lists
\usepackage{verbatim} %useful for something 
\usepackage{xcolor}
\usepackage{setspace}
\usepackage{titlesec}
\usepackage{blindtext} % I have no idea what this is 
\usepackage{caption}  % need this for unnumbered captions/figures
\usepackage{natbib}
\usepackage{appendix}
\usepackage{tikz}
\usepackage{hyperref}


\hypersetup{
    colorlinks=true,
    linkcolor=blue,
    filecolor=magenta,      
    urlcolor=blue,
    pdftitle={Overleaf Example},
    pdfpagemode=FullScreen,
    }

\titleformat{\section}{\bfseries\Large}{Problem \thesection:}{5pt}{}

\begin{document}

\title{AM 216 - Stochastic Differential Equations: Assignment 8}
\author{Dante Buhl}


\newcommand{\wrms}{w_{\text{rms}}}
\newcommand{\bs}[1]{\boldsymbol{#1}}
\newcommand{\tb}[1]{\textbf{#1}}
\newcommand{\bmp}[1]{\begin{minipage}{#1\textwidth}}
\newcommand{\emp}{\end{minipage}}
\newcommand{\R}{\mathbb{R}}
\newcommand{\C}{\mathbb{C}}
\newcommand{\N}{\mathcal{N}}
\newcommand{\Var}{\text{Var}}
\newcommand{\Cov}{\text{Cov}}
\newcommand{\Bino}{\text{Bino}}
\newcommand{\Norm}{\mathcal{N}}
\newcommand{\erf}{\text{erf}}
%\newcommand{\K}{\bs{\mathrm{K}}}
\newcommand{\m}{\bs{\mu}_*}
\newcommand{\s}{\bs{\Sigma}_*}
\newcommand{\dt}{\Delta t}
\newcommand{\dx}{\Delta x}
\newcommand{\tr}[1]{\text{Tr}(#1)}
\newcommand{\Tr}[1]{\text{Tr}(#1)}
\newcommand{\Div}{\nabla \cdot}
\renewcommand{\div}{\nabla \cdot}
\newcommand{\Curl}{\nabla \times}
\newcommand{\Grad}{\nabla}
\newcommand{\grad}{\nabla}
\newcommand{\grads}{\nabla_s}
\newcommand{\gradf}{\nabla_f}
\newcommand{\xs}{x_s}
\newcommand{\x}{\bs{x}}
\newcommand{\xf}{x_f}
\newcommand{\ts}{t_s}
\newcommand{\tf}{t_f}
\newcommand{\pt}{\partial t}
\newcommand{\pz}{\partial z}
\newcommand{\uvec}{\bs{u}}
\newcommand{\bvec}{\bs{B}}
\newcommand{\nvec}{\hat{\bs{n}}}
\newcommand{\tu}{\tilde{\uvec}}
\newcommand{\B}{\bs{B}}
\newcommand{\A}{\bs{A}}
\newcommand{\jvec}{\bs{j}}
\newcommand{\F}{\bs{F}}
\newcommand{\T}{\tilde{T}}
\newcommand{\ez}{\bs{e}_z}
\newcommand{\ex}{\bs{e}_x}
\newcommand{\ey}{\bs{e}_y}
\newcommand{\eo}{\bs{e}_{\bs{\Omega}}}
\newcommand{\ppt}[1]{\frac{\partial #1}{\partial t}}
\newcommand{\pp}[2]{\frac{\partial #1}{\partial #2}}
\newcommand{\pptwo}[2]{\frac{\partial^2 #1}{\partial #2^2}}
\newcommand{\ddtwo}[2]{\frac{d^2 #1}{d #2^2}}
\newcommand{\DDt}[1]{\frac{D #1}{D t}}
\newcommand{\ppts}[1]{\frac{\partial #1}{\partial t_s}}
\newcommand{\pptf}[1]{\frac{\partial #1}{\partial t_f}}
\newcommand{\ppz}[1]{\frac{\partial #1}{\partial z}}
\newcommand{\ddz}[1]{\frac{d #1}{d z}}
\newcommand{\ppzetas}[1]{\frac{\partial^2 #1}{\partial \zeta^2}}
\newcommand{\ppzs}[1]{\frac{\partial #1}{\partial z_s}}
\newcommand{\ppzf}[1]{\frac{\partial #1}{\partial z_f}}
\newcommand{\ppx}[1]{\frac{\partial #1}{\partial x}}
\newcommand{\ddx}[1]{\frac{d #1}{d x}}
\newcommand{\ppxi}[1]{\frac{\partial #1}{\partial x_i}}
\newcommand{\ppxj}[1]{\frac{\partial #1}{\partial x_j}}
\newcommand{\ppy}[1]{\frac{\partial #1}{\partial y}}
\newcommand{\ppzeta}[1]{\frac{\partial #1}{\partial \zeta}}
\renewcommand{\k}{\bs{k}}
\newcommand{\real}[1]{\text{Re}\left[#1\right]}


\maketitle 
% This line removes the automatic indentation on new paragraphs
\setlength{\parindent}{0pt}

\section{Lambda-rule for stochastic derivatives}
    \begin{proof}
        We begin this section by first taking $X(t) = H(t,W(t))$. 
        \begin{align*}
             dX &= \left(X_t + \frac{1}{2}X_{WW}\right)dt + X_WdW
             \\
             &= \left(2tX + \frac{1}{2}X\right)dt + XdW
        \end{align*}
        We notice that if this were true we must have several conditions with
        also follow with our original assumption. First and foremost, 
        \begin{align*}
            X = X_W &\implies X_{W} = X_{WW}
            \\
            X &\propto e^{W(t)}
            \\
            X_t = 2tX &\implies X \propto e^{t^2}
            \\
            X &= X_0e^{t^2 + W(t)}
        \end{align*}
        This all follows if we take the $\lambda$-chain rule and the original
        SDE to be equivalent. What remains is to resolve the initial condition
        and this simply implies that $X_0 = 2$. 
        \begin{gather*}
            H(t,W(t)) = X(t) = e^{t^2 + W(t)}
        \end{gather*}
    \end{proof}

\section{Time Reversal of an SDE}
    \begin{proof}
        \begin{align*}
            \rho_{\left(X(t)|X(t + dt) = x_1\right)} (x_0) &\propto
            \exp\left(\frac{-1}{2\sigma^2dt}\left[(x_0-x_1)^2 + 2c_1dt(x_0 -
            x_1) + c_2dt(x_0 - x_1)^2 + \ldots\right]\right)
            \\
            \hat{\sigma}^2 = \frac{\sigma^2dt}{1 + c_2dt}, &\quad \beta =
            \frac{c_1dt}{1 + c_2dt}
            \\
            \rho_{\left(X(t)|X(t + dt) = x_1\right)} (x_0) &\propto
            \exp\left(\frac{-1}{2\hat{\sigma}^2}\left[\Delta_x^2 + 2\beta\Delta_x
            + \ldots\right]\right)
            \\
            &\propto \exp\left(\frac{-1}{2\hat{\sigma}^2}\left[\Delta_x^2 + 2\beta\Delta_x
            + \beta^2\right]\right)
            \\
            &\propto \exp\left(\frac{-1}{2\hat{\sigma}^2}\left[\Delta_x +
            \beta\right]^2\right)
            \\
            &\propto  \exp\left(-\frac{1 + c_2dt}{2\sigma^2dt}\left[x_0 - x_1 +
            \frac{c_1dt}{1 + c_2dt}\right]^2\right)
            \\
            \left(X(t) | X(t+dt) = x_1\right) &\sim N\left(\frac{-c_1dt}{1 + c_2dt},
            \frac{\sigma^2dt}{1 + c_2dt}\right)
        \end{align*}
    \end{proof}

\section{Feynman-Kac Formula}
    \begin{enumerate}[label=\roman*)]
        \item Write out the FVP
            \begin{align*}
                0 &= u_t +\frac{1}{2}u_{xx} + xu
                \\
                u(x,T,T) = 1
            \end{align*}
        \item Verify the given solution
            \begin{align*}
                u(x,t,T) &= \exp\left(\frac{(T - t)^3}{6} + (T - t)x\right)
                \\
                u_t &= -\left(x + (T - t)^2/2\right)\exp\left(\frac{(T - t)^3}{6} + (T - t)x\right)
                \\
                u_{xx} &= (T - t)^2\exp\left(\frac{(T - t)^3}{6} + (T - t)x\right)
                \\
                0 &=  \exp\left(\frac{(T - t)^3}{6} + (T -
                t)x\right)\left[-\left(x +
                \frac{(T - t)^2}{2}\right) + \frac{(T - t)^2}{2} + x\right]
                \\
                &= 0
                \\
                u(x,T,T) &= \exp(0) = 1
            \end{align*}
            Thus this solution satisfies the FVP. 
        \item This specific problem is easy enough to directly integrate so I
        will do so. 
            \begin{proof}
                We begin by expanding $X(s)$ (Note that, $\int_t^T \hat{W}(s)ds \sim
                N\left(0, \frac{(T - t)^3}{3}\right)$)
                \begin{align*}
                    u(x,t,T) &= E\left[\exp\left(\int_t^T x +
                    \hat{W}(s)ds\right) \Big| \hat{W}(t) = 0\right]
                    \\
                    &= E\left[\exp((T - t)x) \exp\left(\int_t^T
                    \hat{W}(s)ds\right) \Big| \hat{W}(t) = 0\right]
                    \\
                    &= C \exp((T - t)x) \int_{-\infty}^{\infty} e^x
                    e^{-x^2/2\sigma^2}dx
                    \\
                    &= C \exp((T - t)x) \int_{-\infty}^{\infty}
                    e^{-\left(x^2 - 2\sigma^2x\right)/2\sigma^2}dx
                    \\
                    &= C \exp((T - t)x) e^{\sigma^2/2} \int_{-\infty}^{\infty}
                    e^{-\left(x - \sigma^2\right)^2/2\sigma^2}dx
                    \\
                    &= \exp((T - t)x) e^{\sigma^2/2}
                    \\
                    &= \exp\left(x(T-t) + \frac{(T - t)^3}{6}\right) 
                \end{align*}
            \end{proof}
    \end{enumerate}

\section{Integrating Factor to solve SDE}
    \begin{enumerate}[label=\roman*)]
        \item
        \begin{align*}
            dX - \frac{1}{1 + t}X dt &= dW
            \\
            d\left(\frac{X}{1 + t}\right) &= \frac{dW(s)}{1 + s}
            \\
            X(t) &= (1 + t)\left[c +  \int_0^t \frac{dW(s)}{1 + s}\right]
            \\
            X(0) = 1 &\implies c = 1
        \end{align*}
        \item 
            \begin{align*}
                E(X(t)) &= (1 + t)\left(1  + E\left(\int_0^t\frac{dW(s)}{1 +
                s}\right)\right)
                \\
                &= 1 + t
                \\
                \Var(X(t)) &= (1 + t)^2\Var\left(\int_0^t\frac{dW(s)}{1 +
                s}\right)
                \\
                &= (1 + t)^2\left(\int_t^T\frac{ds}{(1 + s)^2}\right)
                \\
                &= (1 + t)^2\left(-\frac{1}{1 + s}\right|_0^t
                \\
                &= (1 + t)^2\left(1 - \frac{1}{1 + t}\right)
                \\
                &= t + t^2
            \end{align*}
    \end{enumerate}

\section{Simulating Feynman-Kac}
    \begin{enumerate}[label=\roman*)]
        \item We can verify this solution by simply plugging it into the
        original PDE and initial condition. 
            \begin{align*}
                u(x,t,T) &= \exp\left(-\frac{(x - t)^2}{2}\right) 
                \\
                u_t &= (x - t)u
                \\
                u_{xx} &= \ppx{}\left(-(x-t)u\right)
                \\
                &= -u + (x-t)^2u
                \\
                0 &= u\left[(x - t) - \frac{1}{2} + \frac{(x-t)^2}{2} -
                \phi\right]
                \\
                &= 0 
            \end{align*}
        \item 
    \end{enumerate}

\section{Nonlinear Function of an RV}

\end{document}
