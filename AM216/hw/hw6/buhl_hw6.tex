\documentclass{article}

\usepackage{graphicx} % Required for inserting images
\usepackage[left=1in,right=1in,top=1in,bottom=1in]{geometry} \usepackage{amsmath}
\usepackage{amsthm} %proof environment
\usepackage{amsthm} %proof environment
\usepackage{amssymb}
\usepackage{amsfonts}
\usepackage{enumitem} %nice lists
\usepackage{verbatim} %useful for something 
\usepackage{xcolor}
\usepackage{setspace}
\usepackage{titlesec}
\usepackage{blindtext} % I have no idea what this is 
\usepackage{caption}  % need this for unnumbered captions/figures
\usepackage{natbib}
\usepackage{appendix}
\usepackage{tikz}
\usepackage{hyperref}


\hypersetup{
    colorlinks=true,
    linkcolor=blue,
    filecolor=magenta,      
    urlcolor=blue,
    pdftitle={Overleaf Example},
    pdfpagemode=FullScreen,
    }

\titleformat{\section}{\bfseries\Large}{Problem \thesection:}{5pt}{}

\begin{document}

\title{AM 216 - Stochastic Differential Equations: Assignment }
\author{Dante Buhl}


\newcommand{\wrms}{w_{\text{rms}}}
\newcommand{\bs}[1]{\boldsymbol{#1}}
\newcommand{\tb}[1]{\textbf{#1}}
\newcommand{\bmp}[1]{\begin{minipage}{#1\textwidth}}
\newcommand{\emp}{\end{minipage}}
\newcommand{\R}{\mathbb{R}}
\newcommand{\C}{\mathbb{C}}
\newcommand{\N}{\mathcal{N}}
\newcommand{\Var}{\text{Var}}
\newcommand{\Cov}{\text{Cov}}
\newcommand{\Bino}{\text{Bino}}
\newcommand{\Norm}{\mathcal{N}}
\newcommand{\erf}{\text{erf}}
%\newcommand{\K}{\bs{\mathrm{K}}}
\newcommand{\m}{\bs{\mu}_*}
\newcommand{\s}{\bs{\Sigma}_*}
\newcommand{\dt}{\Delta t}
\newcommand{\dx}{\Delta x}
\newcommand{\tr}[1]{\text{Tr}(#1)}
\newcommand{\Tr}[1]{\text{Tr}(#1)}
\newcommand{\Div}{\nabla \cdot}
\renewcommand{\div}{\nabla \cdot}
\newcommand{\Curl}{\nabla \times}
\newcommand{\Grad}{\nabla}
\newcommand{\grad}{\nabla}
\newcommand{\grads}{\nabla_s}
\newcommand{\gradf}{\nabla_f}
\newcommand{\xs}{x_s}
\newcommand{\x}{\bs{x}}
\newcommand{\xf}{x_f}
\newcommand{\ts}{t_s}
\newcommand{\tf}{t_f}
\newcommand{\pt}{\partial t}
\newcommand{\pz}{\partial z}
\newcommand{\uvec}{\bs{u}}
\newcommand{\bvec}{\bs{B}}
\newcommand{\nvec}{\hat{\bs{n}}}
\newcommand{\tu}{\tilde{\uvec}}
\newcommand{\B}{\bs{B}}
\newcommand{\A}{\bs{A}}
\newcommand{\jvec}{\bs{j}}
\newcommand{\F}{\bs{F}}
\newcommand{\T}{\tilde{T}}
\newcommand{\ez}{\bs{e}_z}
\newcommand{\ex}{\bs{e}_x}
\newcommand{\ey}{\bs{e}_y}
\newcommand{\eo}{\bs{e}_{\bs{\Omega}}}
\newcommand{\ppt}[1]{\frac{\partial #1}{\partial t}}
\newcommand{\pp}[2]{\frac{\partial #1}{\partial #2}}
\newcommand{\pptwo}[2]{\frac{\partial^2 #1}{\partial #2^2}}
\newcommand{\ddtwo}[2]{\frac{d^2 #1}{d #2^2}}
\newcommand{\DDt}[1]{\frac{D #1}{D t}}
\newcommand{\ppts}[1]{\frac{\partial #1}{\partial t_s}}
\newcommand{\pptf}[1]{\frac{\partial #1}{\partial t_f}}
\newcommand{\ppz}[1]{\frac{\partial #1}{\partial z}}
\newcommand{\ddz}[1]{\frac{d #1}{d z}}
\newcommand{\ppzetas}[1]{\frac{\partial^2 #1}{\partial \zeta^2}}
\newcommand{\ppzs}[1]{\frac{\partial #1}{\partial z_s}}
\newcommand{\ppzf}[1]{\frac{\partial #1}{\partial z_f}}
\newcommand{\ppx}[1]{\frac{\partial #1}{\partial x}}
\newcommand{\ddx}[1]{\frac{d #1}{d x}}
\newcommand{\ppxi}[1]{\frac{\partial #1}{\partial x_i}}
\newcommand{\ppxj}[1]{\frac{\partial #1}{\partial x_j}}
\newcommand{\ppy}[1]{\frac{\partial #1}{\partial y}}
\newcommand{\ppzeta}[1]{\frac{\partial #1}{\partial \zeta}}
\renewcommand{\k}{\bs{k}}
\newcommand{\real}[1]{\text{Re}\left[#1\right]}


\maketitle 
% This line removes the automatic indentation on new paragraphs
\setlength{\parindent}{0pt}

\section{Adjoint of a differential operator}
    \begin{proof}
        \begin{align*}
            \int u L^*[v]dx &= \int L[u]vdx
            \\
            &= \int \left(b(x) u_x + \frac{1}{2} a u_{xx} \right) v dx
            \\
            &= \int vbu_x + \frac{1}{2} avu_{xx} dx
            \\
            &= \left(bvu + \frac{1}{2}avu_x\right)\Big|_{D} - \int \left(u(bv)_x
            + \frac{1}{2}u_x(av)_x \right) dx
            \\
            &= \frac{1}{2} u(av)_x\Big|_D- \int \left(u(bv)_x
            -  \frac{1}{2}u(av)_{xx} \right) dx
            \\
            L_z^*[\star] &= -\left(b\star\right)_z + \frac{1}{2}\left(a\star\right)_{zz}
        \end{align*}
    \end{proof}

\section{Backward Equation with Ito's interpretation}
    \begin{enumerate}[label=\roman*)]
    \item
        \begin{align*}
            u(x,t) &= E(u + dXu_x - dtu_t + dX^2u_{xx}/2)
            \\
            &= u + b(x)dtu_x - u_tdt (a(x)/2)u_{xx}
            \\
            u_t &= b(x)u_x + \frac{a(x)}{2}u_{xx}
        \end{align*}
    \item We have the following boundary conditions which represent the exist
    condition and the initial condition which represents whether the exit
    condition is satisfied at $t = 0$
        \begin{align*}
            u(0,t) &= 1
            \\
            u(L,t) &= 1
            \\
            u(x,0) &= \begin{cases} 0 & 0 < x < L \\ 1 & x \in \{0,
            1\}\end{cases}
        \end{align*}
    \item We begin with the definition of expectation
        \begin{align*}
            E(T) &= \int_0^{\infty} x \Pr(T = x) dx
            \\
            &= \int_0^{\infty} x (\lim_{dt \to 0} \Pr(T < x) - \Pr(T < x -
            dt)) dx
            \\
            &= \lim_{dt \to 0} \int_0^{\infty} x (u(x,t) - u(x, t - dt)) dx
            \\
            & \approx
            \int_0^{\infty} xu_t(x,t)dx
        \end{align*}
    \end{enumerate}
\section{Solve the BVP}
    We begin with the method of integrating factor
    \begin{align*}
        m(x) &= e^{2bx}
        \\
        mT_{xx} + 2bmT_x &= -2m
        \\
        T_x &=\frac{1}{m} \int -2m dx
        \\
        T_x &= -\frac{1}{b} + c_1m^{-1}
        \\
        T_x(L_1) = 0 &\implies c_1 = \frac{m(L_1)}{b}
        \\
        T(x) &= -\frac{x}{b} - \frac{1}{2b^2}e^{2b(L_1 - x)} + c_2
        \\
        T(L_2) = 0 &\implies c_2 = \frac{L_2}{b} + \frac{1}{2b^2}e^{2b(L_1 -
        L_2)}
        \\
        T(x) &= \frac{1}{b}\left( L_2 - x\right) + \frac{1 - e^{2b(L_2 - x)}}{2b^2 e^{2b(L_2 -
        L_1)}}
    \end{align*}

\section{Solving the backward equation with TPD}
    \begin{enumerate}[label=\roman*)]
        \item 
            \begin{align*}
                u(x,t) &= E(A)
                \\
                &= E(E(A|X(T - t + dt) = z + dX))
                \\
                &= E(u(z + dX, t - dt))
                \\
                &= E(u + dXu_z - dtu_t + \frac{1}{2}u_{zz}dX^2)
                \\
                &= u - zu_zdt - u_tdt + \frac{1}{2}dtu_{zz}
                \\
                u_t &= -zu_z + \frac{1}{2}u_{zz}
            \end{align*}
        \item we can find an analytical expression for $u$ since we know exactly
            how $X(T)$ is distributed given $X(T - t) = z$. We have, 
            \begin{align*}
                (X(T) | X(T - t) = z) &\sim  N\left(e^{-t}z, \frac{1}{2}(1 -
                e^{-2t})\right)
                \\
                u(x,t;c_0) &= \frac{1}{\sqrt{\pi(1 - e^{-2t})}}\int_{-\infty}^{\infty} H(x - c)e^{-\frac{\left(x -
                e^{-t}z\right)^2}{1 - e^{-2t}}} dx
                \\
                &= \frac{1}{\sqrt{\pi(1 - e^{-2t})}}\int_{c_0}^{\infty} e^{-\frac{\left(x -
                e^{-t}z\right)^2}{1 - e^{-2t}}} dx
                \\
                &= \frac{1}{2} - \frac{1}{2}\erf\left(\frac{x - e^{-t}z}{\sqrt{1
                - e^{-2t}}}\right)
            \end{align*}
    \end{enumerate}

\section{Linear scalings in an SDE}
    \begin{enumerate}[label=\roman*)]
        \item We begin with the necessary substitutions
            \begin{align*}
                 \frac{1}{a}d\tilde{X}(\tilde{t}) &= dX(t)
                 \\
                 &= -\frac{\mu}{ab}\tilde{X}d\tilde{t} +
                 \sqrt{\frac{\sigma^2}{b}}dW(\tilde{t})
                 \\
                 d\tilde{X}(\tilde{t}) &= -\frac{\mu}{b}\tilde{X}d\tilde{t} +
                 \sqrt{\frac{a^2\sigma^2}{b}}dW(\tilde{t})
                 \\
                 b = \mu \quad & \quad a = \sqrt{\frac{\mu}{\sigma^2}}
            \end{align*}
        \item
            \begin{align*}
                u(x,t;c_0) &=
                \frac{1}{a}u^{(s)}(\tilde{x},\tilde{t};\tilde{c_0})
                \\
                &= \sqrt{\frac{\sigma^2}{\mu}} u^{(s)}(\tilde{x},\tilde{t};\tilde{c_0})
            \end{align*}
      \end{enumerate}

\section{More linear scalings in an SDE}
    \begin{enumerate}[label=\roman*)]
        \item We begin with the necessary substitutions
            \begin{align*}
                 \frac{1}{a}d\tilde{X}(\tilde{t}) &= dX(t)
                 \\
                 &= -\frac{\mu}{a^3b}\tilde{X}^3d\tilde{t} +
                 \sqrt{\frac{\sigma^2}{b}}dW(\tilde{t})
                 \\
                 d\tilde{X}(\tilde{t}) &= -\frac{\mu}{a^2b}\tilde{X}d\tilde{t} +
                 \sqrt{\frac{a^2\sigma^2}{b}}dW(\tilde{t})
                 \\
                 b = \sqrt{\mu}\sigma \quad & \quad a^2 = \frac{\sqrt{\mu}}{\sigma}
            \end{align*}
        \item
            \begin{align*}
                u(x,t;c_0) &=
                \frac{\alpha_0}{a^2}u^{(s)}(\tilde{x},\tilde{t};\tilde{c_0})
                \\
                &= \frac{\alpha_0\sigma}{\sqrt{\mu}} u^{(s)}(\tilde{x},\tilde{t};\tilde{c_0})
            \end{align*}
      \end{enumerate}

\section{Periodic Steady State solution of the forward equation}


\end{document}
